\documentclass[12pt,english]{article}
\usepackage[T1]{fontenc}
\usepackage{mathptmx}
%\renewcommand{\familydefault}{\sfdefault}
\usepackage[latin9]{inputenc}
\usepackage{multirow}
\usepackage{multicol}
\usepackage[letterpaper]{geometry}
\geometry{verbose,tmargin=2cm,bmargin=2cm}
\usepackage{float}
\usepackage{amstext}
\usepackage{amsmath}
\usepackage{mathptmx}
\usepackage{amsthm}
\usepackage{array}
\usepackage[capposition=top]{floatrow}

\newtheorem{hypothesis}{Hypothesis}
\usepackage[title,titletoc]{appendix} 
\usepackage{multirow}
\newtheorem{result}{Result}
%\usepackage{theorem}
\newtheorem{hypo}{Hypothesis}
\usepackage{harvard}
\usepackage{graphicx}
\usepackage{ amssymb }
\usepackage{url}
\usepackage{epstopdf}
\usepackage{caption}
\usepackage{subcaption}
\usepackage{natbib}

%\usepackage{setspace}
\usepackage{esint}
\doublespacing
\usepackage{babel}
\DeclareMathOperator*{\med}{med}

\hyphenation{parti-cularly}

\begin{document}
%\onehalfspacing


\section{Introduction} 
In uncertain times, cooperation is often necessary to ensure smooth continuation of daily life. Unfortunately, as we have come to know during the current pandemic, which asked for strict social distancing and shuttering of many businesses across many countries, the required level of cooperation is not always easy to achieve. In this paper, we study a conflict scenario, where cooperative behavior, as under the current pandemic, can lead to an improved social, and often individual, outcome. For example, consider an agent who must decide whether to hoard ($H$) or not ($D$) a good or liquid asset. When agents have private idiosyncratic costs for $H$ (shoe leather costs), it is then cheaper for some agents to play $H$ than for others. Given the strategic complementarity of our game, the optimal strategy for most players is to follow the actions of others, which then leads to two possible outcomes: (i) a panic run, where everyone plays $H$, or (ii) a social optimum, where everyone plays $D$. Since players have an incentive to follow the actions of others, precommitment to $D$ by some players can help avoid a panic run, and thus ensure a Pareto superior outcome. 

%In certain conflict scenarios, precommitment to strategy can facilitate coordination and lead to Pareto superior outcomes. For example, consider a situation in which a bank run is possible. Agents are faced with the dilemma of whether to withdraw ($H$) or not ($D$) their savings from the bank. If agents have private idiosyncratic costs for $H$ (shoe leather costs), then it is cheaper for some agents to play $H$ than for others. However,  given that the game has strategic complementarity, the optimal strategy for the majority of agents is to follow the actions of other players, which can lead to two possible outcomes: (i) a bank run, where everyone plays $H$, or (ii) bank solvency, where everyone plays $D$. Precommitment to $D$ by some players can help avoid a bank run by motivating others to keep their savings in the bank. 

To study the role of precommitment in a conflict game of strategic complements, we propose a laboratory experiment which varies the timing of play.  We design three environments, which are characterized by the extent of incomplete information: (i) simultaneous (CGO), first proposed by Baliga and Sj\"ostr\"om (2004, hereafter BS04), (ii) sequential (CGS), where the order of play is exogenously imposed, and (iii) endogenous (CGE), where players self-select the order of play. In the latter environment, when both players select to move first (second) the game becomes simultaneous and equivalent to CGO. When the players select different order of play, then the game becomes sequential and equivalent to CGS. Thus, the CGE treatment can resemble both the simultaneous and the sequential treatments, depending on the self-selected order of play. The CGE treatment exhibits higher levels of strategic uncertainty relative to both CGS and CGO treatments, since in CGE players do not know (i) the action of counterparty ($H$ or $D$), nor (ii) the timing of play (sequential or simultaneous). 

In the CGO game, BS04 propose a Bayesian Nash Equilibrium (BNE) which suggests that player types at or below a unique cutoff strategy will play $D$.\footnote{In contrast to most global games literature (Carlsson and Van Damme, 1993; Morris and Shin, 2005), BS04 and Farrell and Saloner (1985) assume that types are not correlated and receive no signals about counterparty payoff. } Even if players are not intrinsically hoarders, they may end up choosing $H$, because they may be unable to completely rule out the possibility that the opponent is so, or that the opponent thinks his opponent hoards. In our experiment, we elicit cutoff strategies in the CGO as well as in the other treatments using a slider, which results in a rich distribution of choices across treatments. Theoretically, the cutoff strategy should increase the frequency of the social optimal outcome ($DD$) in the CGS game. In this case, the mutual fear of $H$ is removed due to the fact that the second mover can observe the action taken by the first mover. In the experiment, we first elicit the cutoff for all pairs, and then select one as a first mover, whose cutoff choice is binding while the second mover can freely respond to the first mover's play. The experimental results confirm the direction of the predictions. Players adjust their cutoffs so that they are willing to play $D$ more often in CGS treatment relative to the CGO treatment, however, not at the extent predicted by theory under the assumption of risk neutrality.

In the CGE treatment, the key question is whether \textit{Doves} ---who decide to play $D$--- or \textit{Hawks}---who decide to play $H$---move early. If there exists a separating timing strategy in equilibrium, in which the dovish action appears early while the hawkish player waits, then the sequential-move structure may endogenously emerge as an equilibrium in which the likelihood of $DD$ is higher than the CGO treatment. Here, we employ Hamilton and Slutsky (1990)'s extended game with action commitment. A player can move early only by selecting an action which is then committed. Thus, after eliciting the cutoff and informing the player that her action commitment is $H$ or $D$, the player chooses the order of play. 

Our experimental design adds two layers of novelty to related literature: (i) we elicit cutoff strategies across all treatments using a slider, which results in a rich distribution of choices across treatments, and (ii) we design gender balanced sessions to study whether gender differences are important in endogenous timing conflict games. The experimental results show that players move their cutoff strategy according to theory, which means that they are willing to play $D$ more often in CGS treatment relative to the CGO treatment. The results from the CGE treatment suggest that player strategy depends on whether the game is viewed as simultaneous or sequential. If a player chooses to move first and selects a cutoff similar to the one observed in CGO (CGS) environment, then the player is treating the game as simultaneous (sequential). Furthermore, in the CGE treatment gender appears to affect how the game is viewed by players ---as a CGO or a CGS. Our analysis of the CGE treatment shows that women prefer to (i) move first, and (ii) guarantee a minimum payoff by playing $H$, which means that they treat the CGE game as a CGO game. In fact, the cutoff chosen by women in the CGE treatment is similar to the one observed in CGO. Men respond differently than women to the CGE environment. They select a cutoff similar to the one observed in CGS, and have a preference for moving second where the initial strategy choice is non-binding. Therefore, the results of our study suggest that women prefer to have more control of the payoff distribution than men by imposing a definitive lower bound on earnings. Such behavior is also consistent with minimizing risk.

We do not find any statistical difference in the willingness to play $D$, the socially optimal strategy, across gender in the CGO and CGS treatments. Both men and women have the same distribution of cutoffs within each treatment. Thus, we can conclude that gender characteristics do not matter for achieving the social optimum in the CGO and CGS environments. Although the direction of the cutoff differences across CGO and CGS follows the theoretical predictions, both men and women fall short in selecting the Nash Equilibirum (NE) cutoff under the assumption of risk neutrality. This produces a lower joint frequency of $DD$ pairs in the CGS treatment compared to the theoretical prediction. In the CGO treatment, on the other hand, we observe that both men and women are willing to select a higher cutoff strategy relative to NE, which improves social welfare outcomes. 

\section{Related literature}

The present paper is related to the studies of Evdokimov and Garfagnini (2018), and Khan (2017) who also test the predictions of the CGO game first proposed by BS04, and observe higher frequencies of $DD$ play than predicted by NE. These studies do not use a cutoff strategy, which appeared in Duffy and Ochs (2012) and most recently in Van Huyck et al. (2018). The last authors survey participants about how they play the game and find that 72 percent report using a threshold. Similar to the behavior observed in our experiment, Duffy and Ochs (2012) also find that there is a significant variation in cutoff strategies across subjects. 

Strategic uncertainty, which exists in our CGE environment, can act as a barrier to superior outcomes and is common in coordination games (see Van Huyck, et al., 1990). CGS helps resolve this uncertainty, as shown by Weber et al. (2004) who find that it is sufficient for the subjects to know that game is sequential in order to improve welfare. Our work is also related to previous experiments which study cutoff strategies in games of incomplete information with strategic complements.\footnote{For a theoretical description of pre-commitment in games of complete information with strategic complements and/or substitutes, please refer to Eaton (2004).} Brindisi et al. (2014) find that, in an investment game with correlated player types (a joint investment opportunity), endogenous timing improves social welfare. The results indicate that while coordination is higher under endogenous timing, it is still lower than predicted. In the simultaneous-move game, subjects use a cutoff strategy which deviates from theory, and approaches payoff dominance--- similar to what we observe in our CGO treatment. Other games of incomplete information (Van Huyck et al., 2018) and complete information (Heinemann et al., 2004 and Duffy and Ochs, 2012) also report similar findings. However, for some classes of simultaneous-move games of incomplete information, risk dominance becomes an important factor in strategy selection (Cabrales et al., 2007).

According to the experimental results of Heggedal et al. (2018), risk aversion can deter a first mover from committing to an irreversible action. Their study tests the main predictions of Farrell and Saloner (1985), which is closely related to the environment of BS04. In stage one, players simultaneously decide whether to: (i) remain at status quo (known payoffs) or (ii) choose an alternative (risky payoff) which is irreversible. In stage two, only those players who have not yet committed to an action choose between the two alternatives. If no player committed to an action in stage one, then second stage decisions become simultaneous. The results show that, when the risk of failure is low, players are willing to commit and that they follow the equilibrium cutoff strategy. When the risk of failure is high (i.e. a large payoff loss if the other does not follow through) players are less willing to commit and the behavior deviates from the theoretical prediction. These findings are similar to what we observe in our CGS treatment and partially in our CGE treatment.

Lack of cooperation has also been observed in situations where women appear to be more averse to the ``sucker" effect, which occurs when individuals free-ride because they believe that others will as well (see Ingram and Berger, 1977; and Van den Assem et al., 2012). A recent lab-in-the-field experiment in rural India by Gangadharan et al. (2019) finds that women leaders contribute less than they propose in a public goods game, possibly supporting the notion that women fear being the ``sucker." However, according to Babcock et al. (2017) women can overcome the ``sucker" effect in a dynamic volunteer game with strategic substitutes. Their results show that women are willing to volunteer more often than men, although it may be costly. This suggests that when there is less payoff uncertainty, women may volunteer more and thus improve social welfare. However, in our game of strategic complements, resolving the payoff uncertainty by choosing $H$ does not lead to Pareto superior outcomes and in fact makes the CGE game resemble CGO. The preference for safe payoffs by women compared to men is also well documented in Crosetto and Filippin (2017).

\end{document}